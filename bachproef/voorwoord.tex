%%=============================================================================
%% Voorwoord
%%=============================================================================

\chapter*{\IfLanguageName{dutch}{Woord vooraf}{Preface}}%
\label{ch:voorwoord}

%% TODO:
%% Het voorwoord is het enige deel van de bachelorproef waar je vanuit je
%% eigen standpunt (``ik-vorm'') mag schrijven. Je kan hier bv. motiveren
%% waarom jij het onderwerp wil bespreken.
%% Vergeet ook niet te bedanken wie je geholpen/gesteund/... heeft
Hierbij presenteer ik mijn bachelorproef, getiteld 'Tracking van koeien en identificatie van hun gedrag', als afronding van mijn opleiding Toegepaste Informatica aan het Departement IT en Digitale Innovatie. In de periode van februari tot mei 2024 heb ik me verdiept in de toepassing van geavanceerde technologieën voor het monitoren van koeiengedrag in de landbouw.
\newline\newline
De keuze voor dit onderzoeksthema is voortgevloeid uit mijn stage bij ILVO Vlaanderen, waar ik de kans kreeg om mijn theoretische kennis toe te passen in een praktijkgerichte setting. Dit project sloot naadloos aan bij mijn stageactiviteiten, waardoor ik een diepgaande ervaring op kon doen in het veld van technologische innovatie in de agrarische sector. Vanwege de aard van het werk bij ILVO Vlaanderen en de vertrouwelijkheidsvereisten, ben ik niet in staat specifieke code uit mijn project openbaar te maken.
\newline\newline
Mijn dank gaat uit naar ILVO Vlaanderen en mijn co-promotor Jeremie Haumont, die mij de unieke mogelijkheid hebben geboden om mijn bachelorproef binnen de context van mijn stage te realiseren. Zijn voortdurende ondersteuning en praktische adviezen waren van onschatbare waarde voor het formuleren en uitwerken van mijn project.
\newline\newline
Verder ben ik dankbaar voor de begeleiding van mijn promotor, Dhr. T. Desmedt, wiens suggesties essentieel waren voor de succesvolle voltooiing voor het schrijven van mijn onderzoek.
\newline\newline
Tot slot wil ik mijn familie en vrienden en stagecollega's bedanken voor hun begrip en aanmoediging gedurende dit proces.
\newline\newline
Veel leesplezier gewenst.
\newline\newline
Tomma Vlaemynck