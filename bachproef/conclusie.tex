%%=============================================================================
%% Conclusie
%%=============================================================================

\chapter{Conclusie}%
\label{ch:conclusie}

% TODO: Trek een duidelijke conclusie, in de vorm van een antwoord op de
% onderzoeksvra(a)g(en). Wat was jouw bijdrage aan het onderzoeksdomein en
% hoe biedt dit meerwaarde aan het vakgebied/doelgroep? 
% Reflecteer kritisch over het resultaat. In Engelse teksten wordt deze sectie
% ``Discussion'' genoemd. Had je deze uitkomst verwacht? Zijn er zaken die nog
% niet duidelijk zijn?
% Heeft het onderzoek geleid tot nieuwe vragen die uitnodigen tot verder 
%onderzoek?

De locatiebepaling van elke koe via cameraperspectieftransformatie vormt een innovatieve uitbreiding van het systeem. Deze techniek, die geavanceerde algoritmen gebruikt om de GPS-locatie van de koe te bepalen op basis van camerabeelden, verhoogt de nauwkeurigheid door integratie van beeldstabilisatietechnieken. Een belangrijk voordeel van deze methode is dat het de noodzaak elimineert om fysieke sensoren of GPS-apparaten direct op de koeien aan te brengen, wat bijdraagt aan een diervriendelijkere benadering van monitoring.

Deze techniek biedt niet alleen een diervriendelijk alternatief, maar verzekert ook continuïteit en betrouwbaarheid in de locatiedata, aangezien er geen risico is op uitval of defecten van sensoren die normaal gesproken op de koeien bevestigd zouden worden. Bovendien is deze methode bijzonder effectief in agrarische gebieden waar traditionele monitoringstechnieken zoals het gebruik van halsbanden of andere vormen van aan het dier bevestigde tracking niet haalbaar of wenselijk zijn.

Door gebruik te maken van deze cameragebaseerde locatiebepaling, kunnen boeren nauwkeurig de locatie en bewegingen van hun vee volgen zonder de welzijn van de dieren in gevaar te brengen. Dit leidt tot beter weidebeheer en geoptimaliseerde veebewegingsroutes, terwijl tegelijkertijd de gezondheid en het comfort van de koeien worden gewaarborgd. Deze aanpak ondersteunt niet alleen een effectiever beheer van het vee, maar bevordert ook een meer ethisch verantwoorde veehouderij.

