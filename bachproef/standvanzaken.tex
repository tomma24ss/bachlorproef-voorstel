\chapter{\IfLanguageName{dutch}{Stand van zaken}{State of the art}}%
\label{ch:stand-van-zaken}

% Tip: Begin elk hoofdstuk met een paragraaf inleiding die beschrijft hoe
% dit hoofdstuk past binnen het geheel van de bachelorproef. Geef in het
% bijzonder aan wat de link is met het vorige en volgende hoofdstuk.

% Pas na deze inleidende paragraaf komt de eerste sectiehoofding.

% Dit hoofdstuk bevat je literatuurstudie. De inhoud gaat verder op de inleiding, maar zal het onderwerp van de bachelorproef *diepgaand* uitspitten. De bedoeling is dat de lezer na lezing van dit hoofdstuk helemaal op de hoogte is van de huidige stand van zaken (state-of-the-art) in het onderzoeksdomein. Iemand die niet vertrouwd is met het onderwerp, weet nu voldoende om de rest van het verhaal te kunnen volgen, zonder dat die er nog andere informatie moet over opzoeken \autocite{Pollefliet2011}.

% Je verwijst bij elke bewering die je doet, vakterm die je introduceert, enz.\ naar je bronnen. In \LaTeX{} kan dat met het commando \texttt{$\backslash${textcite\{\}}} of \texttt{$\backslash${autocite\{\}}}. Als argument van het commando geef je de ``sleutel'' van een ``record'' in een bibliografische databank in het Bib\LaTeX{}-formaat (een tekstbestand). Als je expliciet naar de auteur verwijst in de zin (narratieve referentie), gebruik je \texttt{$\backslash${}textcite\{\}}. Soms is de auteursnaam niet expliciet een onderdeel van de zin, dan gebruik je \texttt{$\backslash${}autocite\{\}} (referentie tussen haakjes). Dit gebruik je bv.~bij een citaat, of om in het bijschrift van een overgenomen afbeelding, broncode, tabel, enz. te verwijzen naar de bron. In de volgende paragraaf een voorbeeld van elk.

% \textcite{Knuth1998} schreef een van de standaardwerken over sorteer- en zoekalgoritmen. Experten zijn het erover eens dat cloud computing een interessante opportuniteit vormen, zowel voor gebruikers als voor dienstverleners op vlak van informatietechnologie~\autocite{Creeger2009}.

% Let er ook op: het \texttt{cite}-commando voor de punt, dus binnen de zin. Je verwijst meteen naar een bron in de eerste zin die erop gebaseerd is, dus niet pas op het einde van een paragraaf.

In de context van hedendaags onderzoek naar de veehouderij is de integratie van sensortechnologie en kunstmatige intelligentie (AI) van cruciaal belang. 
Deze technologieën spelen een sleutelrol in het verbeteren van zowel de efficiëntie en nauwkeurigheid van veebewaking als het bevorderen van hogere normen van dierenwelzijn en ethische landbouwpraktijken, zoals besproken in bronnen zoals "Livestock Monitoring Systems Using Advanced Technology" van Pasture.io~\autocite{PastureIo}.
\subsection{Achtergrond en Belang van Vee Monitoring}
De transitie van traditionele naar moderne methoden in vee monitoring markeert een significante vooruitgang in de agrarische sector. Terwijl vee monitoring vroeger voornamelijk een arbeidsintensief proces was met visuele observatie en manuele registratie, heeft de introductie van technologische innovaties deze processen efficiënter, nauwkeuriger en kosten-effectiever gemaakt\autocite{ToAgriculture}. Deze technologische vooruitgang is met name merkbaar in de implementatie van Internet of Things (IoT) technologieën, die een geïntegreerde aanpak bieden voor het beheer van vee\autocite{IntuzIoT}.

Een van de voornaamste voordelen van moderne vee monitoring is de aanzienlijke verbetering van dierenwelzijn. Dit wordt bereikt door het vroegtijdig detecteren van ziektes en verwondingen, waardoor preventieve of corrigerende maatregelen kunnen worden genomen voordat deze problemen ernstig worden. Dergelijke monitoring stelt boeren in staat om de fok- en voedingscycli van dieren nauwlettend te volgen, wat resulteert in betere reproductieve prestaties en hogere productiviteit\autocite{ToAgriculture}.

De toegenomen productiviteit is een ander belangrijk voordeel van geavanceerde vee monitoring. Door continu de gezondheid, het gedrag en de locatie van dieren te monitoren, kunnen boeren beter geïnformeerde beslissingen nemen over fokken, voeren en grazen. Dit resulteert niet alleen in hoogwaardigere producten, maar leidt ook tot hogere opbrengsten en verhoogde winsten voor de boerderij\autocite{ToAgriculture}. Bovendien biedt IoT-technologie gedetailleerde real-time gegevens over de dieren, wat essentieel is voor een effectieve veehouderij\autocite{IntuzIoT}.

Het gebruik van technologie in vee monitoring heeft ook geleid tot aanzienlijke kostenbesparingen. Door middel van efficiënte resource management, zoals voeder- en waterbeheer, kunnen boeren geld besparen en tegelijkertijd de duurzaamheid van hun boerderijen verbeteren. Vroege detectie van ziektes helpt ook bij het verminderen van dure dierenartskosten en het verlies van vee\autocite{ToAgriculture}\autocite{IntuzIoT}.

Ondanks deze voordelen zijn er ook uitdagingen en beperkingen verbonden aan het implementeren van technologie-gebaseerde vee monitoring systemen. De kosten en complexiteit van dergelijke systemen, evenals privacy- en beveiligingszorgen, vormen significante hindernissen. Daarnaast is er vaak behoefte aan gespecialiseerde training en deskundigheid om deze systemen effectief te gebruiken en te integreren met bestaande boerderij management systemen\autocite{ToAgriculture}\autocite{IntuzIoT}.

Samenvattend biedt moderne vee monitoring via technologische innovaties zoals IoT talrijke voordelen voor de agrarische sector. Het verbetert niet alleen het dierenwelzijn en de productiviteit van de boerderij, maar helpt ook bij het verminderen van operationele kosten. Echter, de implementatie van dergelijke systemen vereist zorgvuldige afweging van de bijbehorende uitdagingen en kosten.

\subsection{Huidige Technieken voor Diermonitoring}
De afgelopen jaren hebben innovaties in technologie, waaronder de integratie van het Internet of Things (IoT) en machine learning, een revolutie teweeggebracht in de methoden voor diermonitoring in de landbouw. Deze geavanceerde computer- en sensortechnologieën hebben nieuwe mogelijkheden gecreëerd voor het verzamelen en analyseren van gegevens over dieren, waardoor onderzoekers gedragingen kunnen identificeren die belangrijk zijn voor het detecteren van ziektes, het onderscheiden van de emotionele toestand van de dieren, en zelfs het herkennen van individuele dieren\autocite{MDPI}.

IoT-systemen spelen een cruciale rol bij het monitoren van de gezondheid en het milieu van vee. Deze technologieën stellen boeren in staat om real-time gegevens te verzamelen over vitale tekenen zoals hartslag en ademhaling, evenals omgevingsfactoren zoals temperatuur en luchtvochtigheid. Met behulp van deze gegevens kunnen boeren snel reageren op abnormale omstandigheden en ziektes vroegtijdig detecteren\autocite{IntuzIoT}.

Een andere belangrijke techniek in moderne veehouderij is de implementatie van elektronische identificatiesystemen, zoals RFID-tags. Deze systemen bieden een betrouwbare methode om individuele dieren te identificeren en te volgen, wat essentieel is voor effectief veebeheer. RFID-tags werken met radiofrequenties om gegevens te verzenden en zijn beschikbaar in verschillende vormen, waaronder oorlabels en injecteerbare transponders\autocite{IntechOpen}.

Daarnaast heeft de toenemende beschikbaarheid en betaalbaarheid van computers en internet geleid tot een bredere adoptie van deze technologieën in de veehouderij. Computers worden nu gebruikt voor het beheren van boerderijen, het analyseren van gegevens, en het ondersteunen van besluitvormingsprocessen, met name op grotere boerderijen\autocite{IntechOpen}.

Deze moderne technologieën transformeren de manier waarop vee wordt gemonitord en beheerd, wat leidt tot verbeterde dierengezondheid, verhoogde productiviteit en efficiëntere bedrijfsvoering. Door gebruik te maken van sensorfusie, IoT, en elektronische identificatie, zijn boeren nu beter in staat om nauwkeurige en actuele informatie over hun vee te verkrijgen, wat essentieel is voor zowel economische succes als dierenwelzijn.
\subsection{Geavanceerde Detectie Technieken in de landbouw}
In de context van jouw onderzoek naar het gebruik van geavanceerde objectdetectiemethoden voor het monitoren en analyseren van koeiengedrag in de landbouw zijn verschillende technologieën van belang. Few-Shot Learning (FSL) speelt een cruciale rol in situaties waar grote datasets ontbreken. Deze benadering, een belangrijk onderdeel van meta-learning, stelt algoritmen in staat om te leren van een beperkt aantal voorbeelden, waardoor het ideaal is voor de landbouwomgeving. 
Technieken zoals Matching Networks en Prototypical Networks kunnen objecten classificeren in situaties met beperkte gegevens, wat bijzonder nuttig is in de landbouw waar het verzamelen van grote datasets uitdagend kan zijn \autocite{ObjectDetectionFewShotLearning2022}​​.

Het You Look Only Once (YOLO) algoritme biedt real-time objectdetectie en wordt gebruikt voor diverse taken in de landbouw, zoals het monitoren en volgen van vee. YOLO valt op door zijn efficiëntie in het identificeren van objecten in beeldmateriaal en speelt een cruciale rol in het volgen van koeiengedrag, waardoor het een waardevol hulpmiddel is binnen de diepere leeralgoritmen die worden gebruikt voor het detecteren en classificeren van objecten \autocite{Badgujar2024YOLO}​​.

AI-gebaseerde objectdetectie, gebaseerd op diepgaand leren, heeft significante vooruitgang geboekt in functie-extractie, beeldrepresentatie, classificatie en herkenning. Deze technologieën worden steeds vaker toegepast in de landbouw voor taken zoals het monitoren van vee en gewassen. Ze bieden de mogelijkheid om complexe objectdetectoren en scène classificatoren te creëren die zowel laag-niveau beeldkenmerken als hoog-niveau semantische informatie combineren, wat essentieel is voor het begrijpen van complexe landbouwomgevingen \autocite{FrontiersPlantScience2022}​.
\subsection{Relevant Onderzoek en Case Studies}
\newline
Een essentieel aspect van deze technologische vooruitgang zijn de sensortechnologieën, die in twee hoofdcategorieën kunnen worden onderverdeeld: draagbare apparaten en omgevingsensoren. 
Draagbare sensoren, die direct op de dieren worden bevestigd, volgen essentiële fysiologische parameters en gedragingen. 
Deze gegevens zijn onmisbaar voor het begrijpen van de gezondheid en het welzijn van het vee. Omgevingsensoren, daarentegen, monitoren de condities in de nabije omgeving van de dieren. 
Deze sensoren, waaronder videocamera's, thermische beeldsensoren en drones, zijn cruciaal voor het verstrekken van een compleet overzicht van de leefomgeving van het vee, zoals gedetailleerd besproken in "Artificial Intelligence and Sensor Technologies in Dairy Livestock Export: Charting a Digital Transformation" gepubliceerd op MDPI~\autocite{MDPIAIandSensors}.
\newline
AI speelt een cruciale rol bij het verwerken en analyseren van de enorme hoeveelheden data die door deze sensoren worden gegenereerd. 
Toepassingen van AI in de veehouderij zijn veelzijdig en omvatten het identificeren van patronen in vee-gedrag en fysiologische parameters, zoals vermeld in de eerder genoemde studie op MDPI~\autocite{MDPIAIandSensors}.
\newline
Een ander belangrijk concept binnen deze context is Precision Livestock Farming (PLF), dat gebruikmaakt van informatietechnologie voor real-time monitoring en beheer van vee. 
Door de implementatie van geavanceerde technologieën zoals microfluidica en geluidsanalyse, heeft PLF geleid tot aanzienlijke verbeteringen in de nauwkeurigheid van gezondheidsmonitoring, zoals besproken in "Precision Livestock Farming Research: A Global Scientometric Review" gepubliceerd op MDPI~\autocite{MDPIPLF}.
\newline
Recent onderzoek, zoals dat van Fuentes et al.~\autocite{Fuentes2023}, onderzoekt de toepassing van diepleertechnieken in de monitoring van individueel koeiengedrag in gesloten schuren met behulp van CCTV-camera's. 
Deze technologieën bieden nieuwe mogelijkheden om complexe gedragingen te identificeren en categoriseren, wat cruciaal is voor efficiënte veebewaking en het bevorderen van dierenwelzijn. 
Dit onderzoek sluit aan bij eerdere bevindingen over het belang van AI en sensorgebaseerde technologieën in de veehouderij.
\newline
Bovendien belicht de studie "Innovative Tools in Farm Animals’ Early Disease Diagnosis"~\autocite{Animals13780} het gebruik van sensoren en draagbare technologieën voor vroege ziekteopsporing bij vee. 
Deze benaderingen zijn essentieel voor het verbeteren van de efficiëntie en effectiviteit van landbouwbedrijven en bieden mogelijkheden voor tijdige interventies en verbeterd beheer.
\newline
In het artikel "Machine Learning Techniques for Behavioral Analysis of Dairy Cows"~\autocite{ar5iv2021} wordt de toepassing van machine learning-technieken voor het analyseren van gedragspatronen bij melkkoeien besproken. 
Deze methoden openen nieuwe mogelijkheden voor gedetailleerde monitoring en verbetering van dierenwelzijn. 
De integratie van AI-technologieën is van essentieel belang voor het decoderen van complexe datasets van veegegevens en biedt waardevolle inzichten voor het verbeteren van landbouwpraktijken.
\newline
Samengevat illustreren deze onderzoeken de impact van deep learning, biosensoren, en machine learning in de veehouderij. 
Deze technologieën verbeteren niet alleen het veebeheer en het welzijn van de dieren, maar dragen ook significant bij aan duurzamere landbouwpraktijken. 
De combinatie van data van zowel draagbare sensoren als omgevingsensoren, samen met de toepassing van diepe leeralgoritmen, zou kunnen leiden tot een nog uitgebreider begrip van de gezondheid en het welzijn van vee en de verdere ontwikkeling van duurzame veehouderijpraktijken.