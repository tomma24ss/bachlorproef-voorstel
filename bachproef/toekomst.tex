\chapter{\IfLanguageName{dutch}{Mogelijk toekomstige uitbreidingen}{Future Research Possibilities}}
\label{ch:toekomst}
\subsection{Doelstelling}
Identificeren van potentiële gebieden voor verdere ontwikkeling en verbetering van de technologie, met als doel de nauwkeurigheid en functionaliteit van het koeiengedrag-detectiesysteem te verhogen.

\subsection{Geavanceerde Deep Learning Models voor Gedragsclassificatie}
Een kerngebied voor toekomstig onderzoek is de ontwikkeling van een geavanceerd diepgaand leermodel dat gedrag classificeert op basis van keypoints. Dit zou een verfijndere en contextueel nauwkeurigere analyse van koeiengedrag bieden, vergeleken met de huidige methoden die voornamelijk gebaseerd zijn op eenvoudige metingen van lichaamshoeken.

\subsection{Integratie van meerdere camera's}
Een andere significante verbetering zou de implementatie van meerdere camera's betreffen. Dit zou een breder en gedetailleerder beeld van de veeomgeving opleveren, waardoor het systeem effectiever koeien op grotere afstanden kan analyseren. Bovendien zou dit helpen bij het begrijpen van de sociale interacties en de ruimtelijke dynamiek binnen kuddes.

\subsection{Ontwikkeling van een Gespecialiseerde Keypoint Dataset}
Het creëren van een gespecialiseerde dataset met keypoints specifiek voor de koeien in de studie is een ander potentieel onderzoeksgebied. Dit zou de afhankelijkheid van algemene voorgetrainde modellen verminderen en de nauwkeurigheid van pose estimation significant verbeteren door deze te optimaliseren voor specifieke kenmerken van de onderzochte koeien.

\subsection{Uitbreiding van Gedragsclassificaties}
Het uitbreiden van gedragsclassificaties om nieuwe categorieën zoals sociale interacties en gezondheidsindicatoren te omvatten, zou een rijker begrip van koeienwelzijn bieden. Dit zou waardevolle inzichten verschaffen die gebruikt kunnen worden om veehouderijpraktijken te verbeteren, niet alleen door de fysieke, maar ook door de sociale en gezondheidsaspecten van koeien te adresseren.

Deze toekomstige onderzoeksrichtingen beloven niet alleen de technologische grenzen van diergedragsanalyse te verleggen maar bieden ook praktische voordelen voor de agrarische sector door een verbeterde monitoring en management van het vee.