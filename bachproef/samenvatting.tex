%%=============================================================================
%% Samenvatting
%%=============================================================================

% TODO: De "abstract" of samenvatting is een kernachtige (~ 1 blz. voor een
% thesis) synthese van het document.
%
% Een goede abstract biedt een kernachtig antwoord op volgende vragen:
%
% 1. Waarover gaat de bachelorproef?
% 2. Waarom heb je er over geschreven?
% 3. Hoe heb je het onderzoek uitgevoerd?
% 4. Wat waren de resultaten? Wat blijkt uit je onderzoek?
% 5. Wat betekenen je resultaten? Wat is de relevantie voor het werkveld?
%
% Daarom bestaat een abstract uit volgende componenten:
%
% - inleiding + kaderen thema
% - probleemstelling
% - (centrale) onderzoeksvraag
% - onderzoeksdoelstelling
% - methodologie
% - resultaten (beperk tot de belangrijkste, relevant voor de onderzoeksvraag)
% - conclusies, aanbevelingen, beperkingen
%
% LET OP! Een samenvatting is GEEN voorwoord!

%%---------- Nederlandse samenvatting -----------------------------------------
%
% TODO: Als je je bachelorproef in het Engels schrijft, moet je eerst een
% Nederlandse samenvatting invoegen. Haal daarvoor onderstaande code uit
% commentaar.
% Wie zijn bachelorproef in het Nederlands schrijft, kan dit negeren, de inhoud
% wordt niet in het document ingevoegd.

\IfLanguageName{english}{%
\selectlanguage{dutch}
\chapter*{Samenvatting}
\lipsum[1-4]
\selectlanguage{english}
}{}

%%---------- Samenvatting -----------------------------------------------------
% De samenvatting in de hoofdtaal van het document

\chapter*{\IfLanguageName{dutch}{Samenvatting}{Abstract}}

Deze bachelorproef focust zich op de toepassing van geavanceerde technologieën voor het monitoren en analyseren van koeiengedrag in de agrarische sector. De inspiratie voor dit onderzoek komt voort uit de behoefte aan efficiëntere en nauwkeurigere methoden voor het volgen van vee, wat essentieel is voor zowel het welzijn van de dieren als voor de optimalisatie van de landbouwpraktijken.

De kern van dit onderzoek is het toepassen van technologieën zoals objectdetectie, keypoints en diepteschatting voor het nauwkeurig volgen en identificeren van specifieke gedragingen van koeien. Deze technologieën bieden mogelijkheden voor precisielandbouw, waarmee de efficiëntie en nauwkeurigheid van het monitoren van vee verbeterd kunnen worden.

De methodologie van dit onderzoek omvatte een uitgebreide literatuurstudie, gegevensverzameling en de evaluatie van verschillende technologische benaderingen. Deze methoden werden toegepast om te bepalen hoe effectief deze technologieën kunnen worden ingezet in een agrarische setting.

De resultaten van dit onderzoek wijzen op een aanzienlijke verbetering in het vermogen om koeiengedrag nauwkeurig te volgen en te interpreteren, wat essentieel is voor effectief veebeheer. Door het verbeteren van de tracking en gedragsidentificatie van koeien, biedt dit onderzoek belangrijke inzichten voor veehouders en technologiebedrijven in de agrarische sector.

Deze studie draagt bij aan de agrarische sector door niet alleen het welzijn van de dieren te verbeteren, maar ook door te ondersteunen in de overgang naar meer geavanceerde en slimme landbouwpraktijken. De bevindingen suggereren dat de geïmplementeerde technologieën potentie hebben voor bredere toepassingen binnen de sector, wat aanzienlijke voordelen kan opleveren voor het efficiënt en effectief beheer van vee.
