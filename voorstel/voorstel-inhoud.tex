%---------- Inleiding ---------------------------------------------------------

\section{Introductie}%
\label{sec:introductie}

Deze bachelorproef verkent de mogelijkheden van geavanceerde technologieën voor het volgen en begrijpen van het gedrag van koeien. De noodzaak van nauwkeurige tracking en gedragsidentificatie in de veehouderij is cruciaal voor zowel dierenwelzijn als operationele efficiëntie. Het onderzoek focust op het analyseren van verschillende technologische methoden, zoals objectdetectie en diepteschatting, om koeien te lokaliseren en hun gedragingen zoals grazen, wandelen en rusten te identificeren. Het doel is om de uitdagingen en kansen van deze technologieën in de praktijk te onderzoeken en te demonstreren hoe ze bijdragen aan verbeterde veehouderijpraktijken.

%---------- Stand van zaken ---------------------------------------------------

\section{State-of-the-art}%
\label{sec:state-of-the-art}

In de hedendaagse veehouderij is technologie cruciaal voor het volgen van koeien en het analyseren van hun gedrag. GPS- en RFID-technologieën worden vaak gebruikt voor basis tracking van vee. Studies zoals 'Advances in GPS Tracking Technology for Large Livestock Management' hebben de effectiviteit van GPS-systemen benadrukt, terwijl 'RFID in Animal Tracking: Applications and Innovations' de rol van RFID-tags in het volgen van vee onderzoekt.

Recentere ontwikkelingen in computer vision en deep learning hebben geleid tot geavanceerdere methoden. Het artikel 'Multi-Feature Tracking Algorithms for Real-time Cattle Monitoring' in Scientific Reports beschrijft een systeem dat deep learning en computervisie combineert om koeien in real-time te volgen in stalomgevingen. Dit toont aan dat geavanceerde technologieën zoals objectdetectie en keypoints-analyse, die verder worden onderzocht in 'Deep Learning in Livestock Behavior Analysis: Trends and Challenges', aanzienlijke mogelijkheden bieden voor gedetailleerde gedragsanalyse.

Diepteschattingstechnieken, besproken in '3D Imaging and Analysis in Livestock Monitoring Systems', voegen een extra laag van precisie toe aan het volgen van koeien. Echter, zoals onderzocht in 'Challenges in Livestock Monitoring: From Traditional to Advanced Technologies', zijn er uitdagingen, zoals het identificeren van gedrag onder verschillende omgevingscondities.

Dit onderzoek bouwt voort op deze bevindingen door de mogelijkheden van geïntegreerde systemen te onderzoeken die zowel de locatie als het gedrag van koeien in diverse situaties kunnen volgen, een gebied dat tot nu toe onderbelicht is in de literatuur.
% Voor literatuurverwijzingen zijn er twee belangrijke commando's:
% \autocite{KEY} => (Auteur, jaartal) Gebruik dit als de naam van de auteur
%   geen onderdeel is van de zin.
% \textcite{KEY} => Auteur (jaartal)  Gebruik dit als de auteursnaam wel een
%   functie heeft in de zin (bv. ``Uit onderzoek door Doll & Hill (1954) bleek
%   ...'')

%---------- Methodologie ------------------------------------------------------
\section{Methodologie}%
\label{sec:methodologie}

\subsection{Literatuurstudie}
\begin{itemize}
  \item \textbf{Doel:} Een grondige review van relevante literatuur om bestaande technieken en uitdagingen in vee tracking te begrijpen.
  \item \textbf{Aanpak:} Analyse van literatuur over GPS, RFID, computervisie en deep learning.
  \item \textbf{Deliverable:} Een literatuuroverzicht dat de experimentele aanpak ondersteunt.
\end{itemize}
\subsection{Datacollectie}
\begin{itemize}
  \item \textbf{Doel:} Verzamelen van real-time data van koeiengedrag onder verschillende omstandigheden via camera's en sensoren op een ILVO boerderij.
  \item \textbf{Aanpak:} Implementatie van high-definition camera's en sensoren. Ontwikkelen van een API voor efficiënte data-extractie en -beheer. Opzetten van data pipelines voor het automatiseren van gegevensverzameling en -verwerking. Gebruik maken van bestaande ILVO datasets voor aanvullende data.
  \item \textbf{Deliverable:} Een rijke dataset van videobeelden en sensorgegevens, inclusief reeds verzamelde data van Ilvo.
\end{itemize}
\subsection{Technologieën en Technieken}
\begin{itemize}
  \item \textbf{Doel:} Het selecteren en integreren van geschikte technologieën voor gedragsanalyse.
  \item \textbf{Aanpak:} Implementatie of aanpassen computervisie technieken zoals objectdetectie en keypoints-analyse, aangepast aan bestaande modellen en systemen die ILVO mogelijk al heeft ontwikkeld.
  \item \textbf{Deliverable:} Een geïntegreerd systeem voor het monitoren van koeiengedrag.
\end{itemize}
\subsection{Data-analyse en -verwerking}
\begin{itemize}
  \item \textbf{Doel:} Uitgebreid analyseren en categoriseren van koeiengedragingen uit de verzamelde data.
  \item \textbf{Aanpak:} Toepassing van geavanceerde machine learning en patroonherkenning algoritmen voor diepgaande analyse. Deze fase omvat het pre-processen van data, trainen van modellen, en het valideren van resultaten.
  \item \textbf{Deliverable:} Een uitgebreid analyserapport met gedetailleerde inzichten in koeiengedrag en bewegingspatronen. Deze fase zal de meeste tijd in beslag nemen, gezien de complexiteit van de data en de vereiste nauwkeurigheid van de analyses.
\end{itemize}
\subsection{Evaluatie van Technologieën}
\begin{itemize}
  \item \textbf{Doel:} Beoordelen van de effectiviteit van de gebruikte technologieën.
  \item \textbf{Aanpak:} Analyseren van nauwkeurigheid, betrouwbaarheid, en kosteneffectiviteit.
  \item \textbf{Deliverable:} Evaluatierapport met aanbevelingen voor technologiegebruik in de landbouw.
\end{itemize}
\subsection{Uitdagingen en Oplossingen}
\begin{itemize}
  \item \textbf{Doel:} Identificeren en aanpakken van mogelijke obstakels in het onderzoek.
  \item \textbf{Aanpak:} Ontwikkelen van strategieën voor omgevingsvariabelen en technische beperkingen.
  \item \textbf{Deliverable:} Een document met geïdentificeerde uitdagingen en voorgestelde oplossingen.
\end{itemize}
\subsection{Validatie en Verfijning}
\begin{itemize}
  \item \textbf{Doel:} Verfijnen van het systeem voor verbeterde nauwkeurigheid.
  \item \textbf{Aanpak:} Aanpassingen op basis van verzamelde data en feedback.
  \item \textbf{Deliverable:} Een geoptimaliseerd tracking- en gedragsanalyse systeem.
\end{itemize}
%---------- Verwachte resultaten ----------------------------------------------
\section{Verwacht resultaat, conclusie}%
\label{sec:verwachte_resultaten}

In dit onderzoek verwachten we de volgende resultaten te verkrijgen:
\begin{itemize}
  \item Verbeterde nauwkeurigheid in het monitoren van koeiengedrag door geavanceerde technologieën, zoals objectdetectie en keypoints-analyse, toe te passen.
  \item Efficiëntere methoden voor gedragsanalyse van koeien in verschillende omgevingsomstandigheden.
  \item Een geïntegreerd systeem dat zowel de locatie als het gedrag van koeien kan volgen en nauwkeurige gegevens kan genereren.
\end{itemize}
Deze resultaten zullen een meerwaarde bieden voor de veehouderij door bij te dragen aan het welzijn van dieren en operationele efficiëntie te verbeteren. Het gebruik van geavanceerde technologieën voor gedragsmonitoring kan leiden tot betere besluitvorming en zorg voor het vee, en uiteindelijk tot positieve economische en ethische gevolgen voor de landbouwsector.

Het is belangrijk op te merken dat de uiteindelijke resultaten kunnen variëren op basis van de uitkomsten van de data-analyse en evaluatie van de gebruikte technologieën. Dit onderzoek zal een grondige analyse bieden om eventuele afwijkingen te verklaren en aanbevelingen te doen voor verdere verbeteringen.
